\documentclass[a4paper, 11pt]{article}
\usepackage[utf8]{inputenc}
\usepackage[catalan]{babel}
\usepackage[left=3cm,right=3cm,top=2cm,bottom=2cm]{geometry}
\usepackage{amsmath, amssymb}
\usepackage{stmaryrd}
\usepackage{hyperref}

\newcommand\en{\raise.5pt\hbox{$\boxempty\kern-4.85pt{}^{\nearrow}$}\kern-2pt}
\renewcommand{\baselinestretch}{1.15}

\hypersetup{
  colorlinks,
  citecolor=blue,
  urlcolor=blue
}

\title{Quantificació de l'efecte d'algunes mesures de lluita contra la Covid-19}
\author{
  Carlo Sala Gancho\\
  Taller de Modelització\\
  Grau de Matemàtiques\\
  Universitat Autònoma de Barcelona
}
\date{Febrer 2021}

\begin{document}
\maketitle
\section*{Enunciat}
  \begin{enumerate}
    \item A molts llocs se'ns diu que per aturar la Covid-19 cal vacunar el 70\% de la població. Però difícilment se'ns explica d'on surt aquest valor concret. Podries explicar-ho? Podem estar segurs que és suficient aquest percentatge? Com depèn el factor de reproducció de la malaltia del percentatge de població immune?
    \item Una altra eina per a lluitar contra la Covid-19 són les mascaretes. Suposem que només usa mascareta un cert percentatge de la població. Suposem també que una mascareta impedeix el pas del virus amb una certa probabilitat. Com influeixen aquests paràmetres en el factor de reproducció de la malaltia? Podries fer una estimació de com hauria evolucionat l'epidèmia si no haguéssim usat mascareta?
  \end{enumerate}
\section{Introducció}
En totes les pandèmies que ha hagut en la història de la humanitat sempre ha quedat el dubte de quants infectats hi ha hagut, realment, d'aquella malaltia. Gràcies als avenços de la tecnologia biomèdica, disposem de proves fiables que permeten saber si una persona està realment infectada, cosa que ens permet avaluar exactament quines són les mesures que permeten reduir al màxim el nombre de contagis. Aquest article pretén valorar, mitjançant un model matemàtic concret si algunes de les mesures preses per frenar l'avançament de la Covid-19 són realment eficaces.\\
  En aquesta recerca ens centrarem en respondre les dues preguntes plantejades a l'apartat anterior, sempre tenint en compte que desenvoluparem un model matemàtic senzill, basat en el model SIR\cite{sir} i que no tenim prou informació per desenvolupar-ne un de més complex, de manera que no podrem donar una resposta extremadament acurada.
\section{Simplificacions adoptades}
  \begin{itemize}
    \item Considerarem que el temps és una variable discreta, amb el dia com a unitat.
    \item Considerarem que la població està distribuida de forma homogènia per tot el territori estudiat.
    \item Considerarem que, a efectes del model, totes les persones del territori estudiat són iguals i tenen els mateixos costums.
    \item Considerarem que només hi ha una única vacuna, amb una eficàcia del 100\%.
    \item Considerarem que només hi ha una única mascareta homologada i l'ús que es fa de la mascareta és el correcte.
    \item Considerarem que la mascareta té la mateixa eficàcia a l'hora de protegir al portador de la mascareta com de protegir al possible contacte.
  \end{itemize}
\section{Model, variables i paràmetres}
  \subsection{Model: SIR}
  En aquesta recerca, com ja havíem avançat, ens basarem en el model SIR.\@ El model SIR\cite{sir, mora} va ser introduit per William O. Kermack i Anderson G. MacKendrick el 1927. Aquest model considera les següents variables d'estat (per evitar que $R$ tingui dos significats diferents al mateix treball, anomenarem la tercera variable $X$):
  \begin{itemize}
    \item[] $S$ \ nombre de persones susceptibles de contraure la malaltia.
    \item[] $I$ \ nombre de persones que poden transmetre la malaltia.
    \item[] $X$ \ nombre de persones que han passat la malaltia (altes o defuncions) i ja no la poden transmetre.
  \end{itemize}
  Aquestes variables evolucionen respecte el temps d'acord amb les equacions diferencials següents, on el producte $\beta S I$ informa del nombre de nous infectats per unitat de temps i el producte $\gamma I$ representa el nombre d'infectats deixen de ser-ho (traspassen o es curen) per unitat de temps, on $\beta$ i $\gamma$ són paràmetres que depenen de cada epidèmia. Llavors:
  \begin{align}
    \label{ed1} dS/dt &= - \beta S I ,\\
    \label{ed2} dI/dt &= \beta S I - \gamma I = (\beta S - \gamma) I,\\
    \label{ed3} dX/dt &= \gamma I
  \end{align}
  Cal fer una petita observació. $d(S+I+X)/dt = 0$, de manera que el total $\mathbf{N} := S + I + X$ és manté constant, i és el nombre d'individus de la societat estudiada.\\
  El factor de reproducció $\mathbf{R}$ defineix el nombre d'infeccions per contacte directe que provoca un infectat, de mitjana. Està relacionat amb els paràmetres $\beta$ i $\gamma$ de la manera següent:~\cite{mora}
  \begin{equation}
    R = \frac{\beta S}{\gamma}
  \end{equation}
  El factor $\mathbf{R_{0}}$, com es pot deduir, és el factor $R$ en el seu estat inicial, és a dir, introduint un subjecte infectat en una societat on no n'hi ha hagut cap.\\
  Podem estimar els paràmetres $\beta$ i $\gamma$ a partir de les dades epidemiològiques de la manera següent, deduint de les equacions~\ref{ed1},~\ref{ed2} i~\ref{ed3}:
  \begin{align}
    \beta S &= \frac{d(I + X)/dt}{I},\\
    \gamma &= \frac{dX/dt}{I},
  \end{align}
  i, per tant,
  \begin{equation}
    R = 1 + \frac{dI/dt}{dX/dt}
  \end{equation}
  Altres variables i paràmetres que usarem a partir d'ara seran:
  \begin{itemize}
    \item Considerarem $\alpha$ el nombre de vacunats respecte de la població total.
    \item Considerarem $m$ el tant per 1 de població que fa servir mascareta.
    \item Considerarem $e$ el tant per 1 d'eficàcia de la mascareta.
    \item $\beta S$ és la taxa de contagi per unitat de temps.
    \item $\gamma$ és la taxa de resolució per unitat de temps.
  \end{itemize}
\section{Pregunta 1. N'hi ha prou amb el 70\% de vacunats?}
  Per entendre si realment n'hi ha prou amb un 70\% de vacunats cal entendre com evoluciona la pandèmia dia rere dia. El factor més important per entendre-ho és $R$, el factor de reproducció. La variació en el nombre de casos es pot definir com una exponencial simple, on la base és $R$ i l'exponent és la variable temps. Per tant, considerant un $t_0$ i un petit increment de temps $\delta$:
  \begin{equation}
    I_{t_{0} + \delta} = I_{t_{0}} R^{\delta} = {\left( \frac{\beta S}{\gamma} \right)} ^{\delta}
  \end{equation}
  Noti's que si $R > 1$, $I$ creix i, si $R < 1$, $I$ decreix.
  Per tant, per aconseguir vèncer la malaltia cal mantenir el factor de reproducció per sota d'$1$. Considerarem l'estat inicial de la pandèmia, on $R$ és constant i igual a $R_0$. Farem servir el valor de $R_0 = 3,28$\cite{jtm}. $R_0$ considera que no hi ha cap subjecte immune.\\
  Notem que $1 - \alpha$ és el tant per 1 de persones susceptibles a la malaltia respecte la població total. Com que $\beta S$ és la taxa de contagi, si disminuim el nombre de persones susceptibles, la taxa de contagi disminuirà, perquè els contactes deixaran de ser efectius per transmetre la malaltia (matemàticament ho veiem perquè $R = \frac{\beta S}{\gamma}$ i, llavors $R$ és directament proporcional a $\beta$). Llavors, definim el factor de reproducció en funció de el nombre de persones vacunades:
  \begin{equation}
    R = R_0 \left( 1 - \alpha \right)
  \end{equation}
  Noti's que $\alpha$ és el valor que busquem, per tant:
  \begin{equation}
    \alpha = 1 - \frac{R}{R_0} = 1 - \frac{1}{3,28} = 0,695
  \end{equation}
  Veiem, per tant que amb un percentatge més gran que 69,5\% de la població vacunada ja aconseguiríem la anomenada \textit{immunitat de grup}\cite{herd}, ja que la malaltia aniria perdent casos fins a desaparèixer.\\
  En efecte, amb una vacunació del 70\% de la població s'aconseguiria frenar la Covid-19 dins del nostre marc teòric. Cal veure, això sí, si el nostre marc teòric s'ajusta a la realitat.
  \subsection{Possibles refinaments}
  Dins del marc teòric hem vist que la vacuna aconseguiria frenar l'avançament de la Covid-19 si es vacuna a un mínim del 70\% de la població però cal entendre les simplificacions que hem agafat.
  \begin{itemize}
    \item Només hem considerat una branca de Covid-19. Per exemple, la branca britànica sembla ser que és més infeccionsa que l'original.\cite{tg}
    \item Hem considerat que la vacuna té un 100\% d'efectivitat.
  \end{itemize}
  També cal considerar que hem fet tota la valoració com si s'hagués vacunat a l'inici de la pandèmia, i no pas ara, que hi ha certa part de la població immunitzada per haver passat la pandèmia i el valor $R$ no es mantindria constant durant tota la campanya de vacunació.
\section{Pregunta 2. És positiu l'ús de la mascareta?}
  Volem veure com afecta l'ús de la mascareta al factor de reproducció $R$ respecte els paràmetres $m$ i $e$.\\
  Sabem que $R = \frac{\beta S}{\gamma}$, on $\beta S$ és la taxa de contagi. Considerem llavors que l'ús de la mascareta faria variar la taxa de contagi, concretament faria variar $\beta$. Per tant, podem definir el paràmetre següent:
  \begin{equation}
    \beta ' = \beta {\left( 1 - m \cdot e \right)}^2,
  \end{equation}
  on $\beta ' S$ és la taxa de contagis usant mascareta. $1 - m \cdot e$ és la probabilitat que una persona no porti mascareta o bé que la mascareta sigui no efectiva en aquell moment. Elevem aquest terme al quadrat ja que hem considerat que un contacte és entre 2 persones. Per facilitar la lectura, considerem $p = {\left( 1 - m \cdot e \right)}^2$.\\
  Així doncs, tenint en compte l'ús de la mascareta, el factor de reproducció de la malaltia es veu afectat de la següent manera:
  \begin{equation}
    R' = \frac{\beta \cdot p}{\gamma} = R \cdot p
  \end{equation}
  Veiem, per tant, que si no haguéssim usat mascareta tots aquests mesos l'evolució de la pandèmia hagués estat diferent. Vegem ara com varien les diferents equacions diferencials entre fer-ne servir o no, tenint en compte l'apartat les definicions que hem fet en apartats anteriors:
  \begin{equation}
    \frac{dS'}{dt} = - \beta \cdot p \cdot S \cdot I \quad \quad \quad \frac{dI'}{dt} = \beta \cdot p \cdot S \cdot I - \gamma \cdot I \quad \quad \quad \frac{dR'_e}{dt} = \gamma \cdot I
  \end{equation}
  Com que $p \leq 1$, llavors $\left| \frac{dS'}{dt} \right| \leq \left| \frac{dS}{dt} \right|$. D'aquí inferim que els susceptibles a contagiar-se disminueixen més lentament i que, per tant, el creixement d'infectats és més lent.\\
  D'altra banda, el fet de tenir el factor $p$ també afecta en la quantitat d'infectats total en el pic de la malaltia, que trobarem quan $\frac{dI}{dt} = 0$ i, per tant, quan $I \left( \beta S - \gamma \right) = 0$, amb $I \neq 0$. Equivalentment, hem de veure quan $\beta S = \gamma$, i $\gamma$ és independent de l'ús o no de mascareta. Vegem-ho:
  \begin{equation}
    \beta' \cdot S' = \beta \cdot S \Longrightarrow \beta \cdot p \cdot S' = \beta \cdot S \Longrightarrow S' \cdot p = S \leq S'
  \end{equation}
  Deduim llavors que la quantitat de susceptibles en el pic és més gran quan usem mascareta que quan no és així i, per tant, la quantitat d'infectats en el aquell moment és més petita quan usem mascareta que quan no n'usem.
  \subsection{Possibles refinaments}
  Dins del marc teòric hem vist que l'ús de mascareta, sigui quina sigui la seva efectivitat, ajuda a millorar les dades de la pandèmia. Tanmateix, podríem refinar la nostra recerca de la següent manera:
  \begin{itemize}
    \item Només hem considerat un tipus de mascareta, que a més té la mateixa eficàcia a l'hora de protegir el portador i el contacte del portador.
    \item Hem considerat que tota la població fa servir la mascareta de forma correcta, i això no sempre és cert.
  \end{itemize}
\section{Conclusió}
  Aquest recerca ha estat summament breu per poder definir i modelar, d'una manera precisa, la pandèmia de la Covid-19.\\
  Tanmateix, en aquest treball hem pogut esvair el dubte del 70\% per respondre que sí, efectivament n'hi ha prou amb un 70\% de població vacunada, tot i haver tingut en compte que cap persona ha adquirit la immunitat d'haver passat la malaltia. Aquest fet ens fa pensar que el percentatge podria ser substancialment inferior i que, en efecte, el nostre marc teòric pot ser traslladat a la realitat.\\
  D'altra banda, hem pogut comprovar que l'ús de la mascareta fa que, en menor o major mesura en funció de l'eficàcia de la mascareta i el percentatge de població que l'usa, es redueixi el nombre de contagis per unitat de temps i que, d'altra banda, el pic de contagis de la pandèmia sigui més petit.
\begin{thebibliography}{99}
  \bibitem{data}Wikipedia. \textsl{COVID-19 pandemic data.} \href{https://en.wikipedia.org/wiki/Template:COVID-19_pandemic_data}{\en}
  \bibitem{sir}Wikipedia. \textsl{Modelo SIR.} \href{https://es.wikipedia.org/wiki/Modelo_SIR}{\en}
  \bibitem{herd}Wikipedia. \textsl{Herd immunity.} \href{https://en.wikipedia.org/wiki/Herd_immunity}{\en}
  \bibitem{mora}Xavier Mora. \textsl{El nombre de reproducció de la COVID-19 i el model SIR.\@ L'efecte dels retards de comptabilització.} 2020. \textit{Universitat Autònoma de Barcelona.} \href{https://mat.uab.cat/web/matmat/wp-content/uploads/sites/23/2020/06/v2020n02.pdf}{\en}
  \bibitem{jtm}Ying Liu, Albert A. Gayle, Annelies Wilder-Smith, Joacim Rocklöv. \textsl{The reproductive number of COVID-19 is higher compared to SARS coronavirus.} \textit{Journal of Travel Medicine.} \href{https://www.researchgate.net/publication/339272432_The_reproductive_number_of_COVID-19_is_higher_compared_to_SARS_coronavirus/fulltext/5e724347a6fdcc37caf4cf91/The-reproductive-number-of-COVID-19-is-higher-compared-to-SARS-coronavirus.pdf}{\en}
  \bibitem{tg}The Guardian. \textsl{New Covid variant with potentially worrying mutations found in UK.} \href{https://www.theguardian.com/world/2021/feb/15/32-cases-of-latest-covid-variant-of-concern-found-in-uk}{\en}
\end{thebibliography}
\end{document}
